
% Schriftart und Zeichendarstellung
\usepackage[utf8]{inputenc}     % Unterstützung für Umlaute
\usepackage[T1]{fontenc}        % Schriftartpaket
\usepackage[ngerman]{babel}     % Neue Rechtschreibung



%% Layout
% Seitenlayout mit geometry
\usepackage{geometry}
\geometry{left=2.5cm,right=2.5cm,bottom=2.5cm,top=2.5cm}
\parindent 0pt % keine Absatzeinrückung
% Kopf- und Fußzeile
\usepackage[automark,headsepline]{scrpage2}
\clearscrheadings
\clearscrplain
\ohead{\headmark}
\ifoot{\tiny\nummer}
\ofoot{\pagemark}
\setheadsepline{.4pt} % Linie unter dem Head
\headheight 18pt
% Anderthalbzeilig

% Symbole
\usepackage{textcomp}           % Mathematische Symbole für den Textmodus
\usepackage{amsfonts}
\usepackage{amssymb}            % Fraktur, doppeltgestrichene Buchstaben
\usepackage[intlimits]{amsmath} % Integralgrenzen werden über und unter das Integral geschrieben
\usepackage{upgreek}            % aufrechte griechische Buchstaben
\usepackage{sistyle}            % für Einheiten
\SIstyle{German}                % Komma statt Punkt als Dezimaltrennzeichen

%\mathindent15mm                 % Linksbündige Formeln

% Grafikpakete
\usepackage[pdftex]{graphicx}   % .pdf, .png, .jpg können direkt in pdftex verwendet werden
\usepackage{rotating}           % Text und Bilder drehen


% Abschnittsnummern in Zahlen einbeziehen
\numberwithin{table}{chapter}
\numberwithin{figure}{chapter}
\numberwithin{equation}{chapter}

% komplexe Matrix- bzw. Vektorschreibweise
\newcommand{\KVM}[1]{\underline{\boldsymbol{#1}}} % komplexer Vektor oder Matrix
\newcommand{\VM}[1]{\boldsymbol{#1}}              % Vektor oder Matrix
\newcommand{\K}[1]{\underline{#1}}                % komplexe Notation
\newcommand{\bez}[1]{\mathrm{#1}}                 % Bezeichner als Index
\newcommand{\imag}{\mathrm{j}}                    % imaginäre Einheit

% Für einen Paragraph mit Text in der nächsten Zeile
\makeatletter
\renewcommand\paragraph{\@startsection{paragraph}{4}{\z@}%
  {-3.25ex\@plus -1ex \@minus -.2ex}%
  {1.5ex \@plus .2ex}%
  {\normalfont\normalsize\bfseries}}
\makeatother

% Abschluss
% Name des Instituts
\newcommand{\institut}{Professur für Mess- und Sensorsystemtechnik}

% Thema der Arbeit
\newcommand{\thema}{Thema der Arbeit}

% Nummer der Arbeit
\newcommand{\nummer}{DA~7/2022}

% Vorname des Bearbeiters (du)
\newcommand{\vorname}{Maximilian}

% Nachname des Bearbeiters
\newcommand{\nachname}{Ehrhardt}

% Geburtsdatum des Bearbeiters
\newcommand{\geburtsdatum}{30.01.1993}

% Geburtsort des Bearbeiters
\newcommand{\geburtsort}{Merseburg}

% Titel und Name des Betreuers
\newcommand{\betreuer}{Dipl.-Ing. Hannes Radner}

% Titel und Name des verantw. Hochschullehrers
\newcommand{\hslehrer}{Prof. Dr.-Ing. habil. J. Czarske}

% Datum der Einreichung
\newcommand{\einreichung}{\today}

% Angabe von URL-Links
\usepackage{hyperref}

%Erstellung von Bildern
\usepackage{pgf}
\usepackage{tikz}
\usetikzlibrary{shapes,arrows,arrows.meta,positioning,automata,calc}