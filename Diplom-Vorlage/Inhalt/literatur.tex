
% Die Bibliographie kann auf zwei Arten realisiert werden:
% ... entweder so:
 \begin{thebibliography}{99}
\bibitem{wissarbeit}Empfehlungen für die Ausarbeitung wissenschaftlicher Arbeiten; Studienarbeiten, Diplomarbeiten, Masterarbeiten; Ausgabe Juni 2010;\\
\footnotesize http://www.et.tu-dresden.de/etit/uploads/media/EmpfehlungWissenschArbeiten2012\_03.pdf
\normalsize
\bibitem{hoffmann}Hoffmann, Rüdiger: Signalanalyse und -erkennung. Springer Verlag. Berlin Heidelberg, 1998
\bibitem{daubechies}Daubechies, Ingrid: Ten lectures on wavelets. Capital City Press. Montpelier Vermont, 1992
\bibitem{kopka}Kopka, Helmut: \LaTeX~Band 1~Einführung. Pearson Studium. 2005
\bibitem{din1303}Norm DIN~1303. \emph{Vektoren, Matrizen, Tensoren -- Zeichen und Begriffe}
\bibitem{din1304}Norm DIN~1304 Teil~3. \emph{Formelzeichen für die Erzeugung, den Transport und die Verteilung elektrischer Energie}
\bibitem{led}http://www.latexeditor.org/
\end{thebibliography}
% ... oder durch Verwendung des BibTeX-Paketes (besser bei vielen Quellen)
% Dazu bitte entsprechende Literatur (z.B.: Kopka) wälzen!