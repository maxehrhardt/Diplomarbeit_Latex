
\chapter{Einleitung}


Störungen bei der Lichtausbreitung können optische Messungen dramatisch verschlechtern oder sogar vollständig verhindern. Beispiele sind fluktuierende Oberflächen von Flüssigkeiten oder Licht streuendes biologisches Gewebe. Mit der Verfügbarkeit moderner adaptiver Optiken, wie deformierbaren Spiegeln und Flüssigkristallmodulatoren besteht die Möglichkeit, solche Lichtverzerrungen zu korrigieren. An der Professur MST wird erforscht, welche Methoden, Regelungsstrategien und Komponenten in optische Messverfahren implementiert werden können, um Messungen auch in Anwesenheit starker Störungen oder tief in biologischem Gewebe zu erlauben. [\url{https://tu-dresden.de/ing/elektrotechnik/iee/mst/forschung/forschungsfelder}]